% !TeX TXS-program:bibliography = txs:///biber

\documentclass{article}
\usepackage[utf8]{inputenc}
\usepackage{etoolbox}
% ======== SET THIS FLAG to indicate whether to use a local or remote bib file
\newtoggle{local}
\togglefalse{local}
%\toggletrue{local}

\usepackage[%
plainpages,%
colorlinks,%
urlcolor=black,%
filecolor=black,%
pdfpagemode=UseOutlines,%
pdfauthor={Lars Vilhuber},%
pdfsubject={Bibliography},%
]{hyperref}
\usepackage[citestyle=authoryear,maxnames=10,backend=biber]{biblatex}

\nocite{*}
\author{lars.vilhuber }
\date{\today}


\iftoggle{local}{
	\addbibresource{Biblio-Bibtex.bib}
	\title{NCRN global Bibliography [local]}
}%else
{
	\addbibresource[location=remote]{https://www.ncrn.info/documents/bibliographies/export/bibtex}
	\title{NCRN global Bibliography}
}

\begin{document}
\maketitle

This bibliography is generated for the NSF-Census Research Network (NCRN) from data made available at \url{http://www.ncrn.info}. Articles are in alphabetical order of the first author's last name.

\printbibliography[type=article,title={Articles}]
\printbibliography[type=report,title={Preprints and Working papers}]
\printbibliography[nottype=report,nottype=article,title={Other publications}]
\newpage
\begin{quote}
	\footnotesize
This bibliography can be easily recreated by cloning our git repository at \url{https://github.com/labordynamicsinstitute/ncrncobib}.
\iftoggle{local}{This version was created from the BIB file in the repository as of \today. A flag can be set in the preamble of this TEX file to download a remote copy from}{The bibliographic database was downloaded on \today directly from} \url{https://www.ncrn.info/documents/bibliographies/export/bibtex}.
\iftoggle{local}{}{A flag can be set in the preamble of this TEX file to use the local checked out version in the git repository.}


\end{quote}
\end{document}
